\documentclass{article}

\usepackage{amsmath}
\usepackage{amssymb}
\usepackage{bm}

\newtheorem{theorem}{Theorem}

\newcommand{\x}{\textbf{x}}
\newcommand{\y}{\textbf{y}}
\newcommand{\X}{\textbf{X}}
\newcommand{\Y}{\textbf{Y}}
\newcommand{\F}{\textbf{F}}
\newcommand{\f}{\textbf{f}}
\newcommand{\U}{\textbf{U}}
\newcommand{\V}{\textbf{V}}
\newcommand{\C}{\textbf{C}}
\newcommand{\B}{\textbf{B}}
\newcommand{\E}{\textbf{E}}
\newcommand{\I}{\textbf{I}}
\newcommand{\R}{\textbf{R}}
\newcommand{\Chi}{\boldsymbol{\chi}}
\newcommand{\del}{\nabla}
\newcommand{\eqdef}{:=}
\newcommand{\curly}[1]{\mathcal{#1}}
\newcommand{\bld}[1]{\textbf{#1}}

\begin{document}

\begin{titlepage}
	\title{Notes on:\\The Mechanics and Thermodynamics of Continua}
	\date{\today}
	\maketitle
\end{titlepage}	
\tableofcontents
\newpage
	
	
\section{Motion of a Body}
$B$ is a reference body containing points $\X$ which are material points.
There is a one-to-one function
$$\x = \Chi(\X,t)$$
taking reference material points $\X$ to spatial points $\x$ at time $t$.
We require $$J(\X,t) \eqdef \det \del \Chi_t(\X) > 0$$
where J is the volumetric Jacobian of the mapping $\Chi_t$ at $\X$.
Region occupied by body $B$ at time $t$ is
$$\curly{B}_t = \Chi_t(B)$$
is the deformed body at time $t$.
		
\subsection{Convection of Sets with the Body} $A$ is a material set. Then \textbf{deforms to} $\curly{A}_t$ at time $t$. 
$\curly{A}_t$ \textbf{convects with the body} if there is a set $A$ of material points such that
$$\curly{A}_t = \Chi_t(A)$$ for all $t$.
Note that material cannot cross the boundary of a spatial set which convects with the body. Also note that if $\X$ is on $\partial B$ (boundary), then $\Chi(\X,t)$ is on $\partial \curly{B}_t$ for all time $t$ and conversely.

\section{The Deformation Gradient}
The \textbf{Deformation gradient} of a body is $$\textbf{F} = \del \Chi, \hspace{1cm} F_{ij} = \frac{\partial \chi_i}{\partial X_j},$$ the Jacobian matrix of $\x = \x(\X)$.
As above
$$J = \det \textbf{F} > 0.$$
\subsection{Approximation of a Deformation by a Homogeneous Deformation}
\subsubsection{Homegeneous Deformations} Fix time $t$ so that $$\Chi(\X) \equiv \Chi_t(\X).$$
$\Chi$ is a \textbf{homogeneous deformation} if $\textbf{F}(\X) \equiv \textbf{F}(\X,t)$ is independent of $\X$. So
$$\Chi(\X) - \Chi(\bld{Y}) = \bld{F}(\X - \bld{Y})$$
for \bld{all} material points $\X$ and $\bld{Y}$. By the above, \underline{$\bld{F}$ maps material vectors to spatial vectors.} Then, also, $\X - \bld{Y} = \bld{F}^{-1}[\Chi(\X) - \Chi(\bld{Y})]$ so that $\bld{F}^{-1}$ maps spatial vectors too material vectors. Taking the inner product with a spatial vector $\bld{s}$ gives
$$\bld{s} \cdot [\Chi(\X) -\Chi(\Y)] = \bld{s} \cdot [\F(\X - \Y)]  = (\F^T\bld{s})\cdot(\X - \Y)$$ so that $\F^T$ maps spatial vectors to material vectors.\\
Summarizing the mapping properties:
\begin{enumerate}
	\item $\F$ and $\F^{-T}$ map material vectors to spatial vectors
	\item $\F^{-1}$ and $\F^T$ map spatial vectors to material vectors
\end{enumerate}

\subsubsection{General Deformations}
Let $\Chi_t$ be an arbitrary deformation. Taylor expanding the deformation about material point $\X$ gives 
$$\underline{\Chi_t(\Y) - \Chi(\X) = \F(\X,t)(\Y - \X)} + o(|\Y - \X) \hspace{1cm} \text{as } |\Y - \X| \to 0.$$ Therefore, $\F(\X,t)(\Y - \X)$ is an approximation of $\Chi_t(\Y) - \Chi(\X)$. Also, since $\X$ is fixed in the Taylor expansion, $\F(\X,t)$ is constant. Thus
the underlined portion is the second definition of a homogeneous deformation. Therefore, \emph{within a neighborhood of a material point $\X$ and to within an error of $o(|\Y -\X|)$, a deformation behaves like a homogeneous deformation.}\\
So with $o(|\Y - \X|)$ small, we have:
\begin{enumerate}
	\item (M1) $F(\X,t)$ can be thought of as a mapping of an infinitesimal neighborhood of $\X$ in the reference body to an infinitesimal neighborhood of $\x - \Chi_t(\X)$ in the deformed body.
	\item (M2) This gives an asymptotic meaning to the formal relation
	$$\bld{dx} = \F(\X,t)\bld{dX}$$
\end{enumerate}
Now, we have that the mapping properties for a homogeneous deformation hold pointwise for the deformation gradient in an arbitrary deformation. For example, for a given $\X$, the linear transformation $\F(\X,t)$ associates with each material vector $\bld{m}$ a spatial vector $\bld{s} = \F(\X,t)\bld{m}$.

\subsection{Convection of Geometric Quantities} 
\subsubsection{Infinitesimal Fibers}Define the temporally constant material vector field $\textbf{f}_R$ associated with a given spatial vector file $\textbf{f}$ by
\[
\textbf{f}(\x,t) = \F(\X,t)\textbf{f}_R(\X) \hspace{1cm} \x = \Chi_t(\X) \tag{6.8} \label{eq:6.8}
\]
 for all $\X$ and $t$.

Now by the above statements about the local homogeneity of deformation, we can see 
equation~\ref{eq:6.8} becomes
\[
\epsilon\textbf{f}(\x,t) = \F(\X,t)(\epsilon\textbf{f}_R(\X)) \tag{6.9} \label{eq:6.9}
\] 
for $\epsilon > 0$. This can be considered as describing the local deformation when the neighborhood of $\X$ under consideration is magnified by a factor of $\epsilon^{-1}$.\\

In equation~\ref{eq:6.8}, $\f_R(\X)$ is an \textbf{infinitesimal undeformed fiber} and $\f(\x,t) = \F(\X, t)\f_R(\X)$ is the corresponding \textbf{(infinitesimal) deformed fiber}. We can see the deformed fiber as \textbf{embedded} in the deforming body $\curly{B}_t$ and we say $\f(\x,t)$ convects with the body.\\

\textbf{f convects with the body} and \textbf{f is convecting} mean that there is a fixed (time independent) material vector field $\f_R$(\X) such that equation~\ref{eq:6.8} holds.\\

\subsubsection{Curves}
$C$ is a \textbf{material curve} with parameterization $\hat{\X}(\lambda)$, $\lambda\in [\lambda_0,\lambda_1]$ which does not intersect itself. The corresponding \textbf{spatial curve} is $\curly{C}_t = \Chi_t(C)$. Note the time-dependent parameterization. Then $\curly{C}_t$ is a curve \textbf{embedded} in the deforming body.

\subsubsection{Tangent Vectors}
Given $\X$ on $C$, the tangent to $C$ at $\X$ is
\[
\boldsymbol{\tau}_R(\X) = \frac{d\hat{\X}(\lambda)}{d\lambda} \tag{6.12} \label{eq:6.12}
\]
Then the corresponding tangent to $\curly{C}_t$ at $\x$ is
\[
\boldsymbol{\tau}(\x,t)= \frac{\partial \hat{\x}_t(\lambda)}{\partial\lambda} \tag{6.13}\label{eq:6.13}
\]
which gives
\[
\boldsymbol{\tau}(\x,t) = \F(\X,t)\boldsymbol{\tau}_R(\X) \tag{6.15}\label{eq:6.15}
\]


\begin{theorem}[Transformation Law for Tangent Vectors]
	At each time, the relation (\ref{eq:6.15}) associates with any vector $\boldsymbol{\tau}_R$ at $\X$ a vector $\boldsymbol{\tau}$ at $\x = \Chi_t(\X)$ with the following property: if $\boldsymbol{\tau}_R$ is tangent to a material curve at $\X$, then $\boldsymbol{\tau}$ is tangent to the corresponding deformed curve through $\x$.
\end{theorem}

\subsubsection{Bases}
Fix a material basis
$$\lbrace \textbf{m}_i(\X)\rbrace = \lbrace \textbf{m}_1(\X),\textbf{m}_2(\X),\textbf{m}_3(\X) \rbrace.$$
Then the associated spatial basis is
\[
\lbrace \textbf{s}_i(\x,t)\rbrace = \lbrace \F(\X,t)\textbf{m}_i(\X)\rbrace \tag{6.16}\label{eq:6.16}
\]

at $\x = \Chi(\X, t)$.
This spatial basis convects with the body, ie is embedded in the deforming body.

\section{Stretch, Strain, and Rotation}
\subsection{Stretch and Rotation Tensors. Strain}
The polar decomposition (rotation $\textbf{R}$ and positive-definite symmetric tensors \textbf{U} and \textbf{V}) of the deformation gradient:
\[
\F = \textbf{R}\textbf{U} = \textbf{V}\textbf{R} \tag{7.1}\label{eq:7.10}
\]

\textbf{U} is the \textbf{right stretch tensor} and \textbf{V} is the \textbf{right stretch tensor}.

The following is good for theoretical but difficult in application
\[
\centering
\textbf{U} = \sqrt{\F^T\F} \hspace{2cm} \text{and} \hspace{2cm}\textbf{V} = \sqrt{\F\F^T} \tag{7.2} \label{eq:7.2}
\]

\textbf{Left and Right Cauchy-Green (deformation) tensors C and B}:
\[
\begin{array}{lr}
\textbf{C} = \textbf{U}^2 = \F^T\F,  &  C_{ij} = F_{ki}F_{kj} = \frac{\partial \Chi_k}{\partial\X_i}\frac{\partial\Chi_k}{\partial\X_j}\\

\textbf{B} = \textbf{V}^2 = \F\F^T,  &  B_{ij} = F_{ik}F_{jk} = \frac{\partial \Chi_i}{\partial\X_k}\frac{\partial\Chi_j}{\partial\X_k}
\end{array} \tag{7.3}\label{eq:7.3}
\]

Then,
\[
\textbf{V} = \textbf{RUR}^T \hspace{1cm} and \hspace{1cm} \textbf{B} = \textbf{RCR}^T \tag{7.4} \label{eq:7.4}
\]

and
\[
\text{\U, \V, \C, and \B\space are symmetric and positive-definite.}\tag{7.5}\label{eq:7.5}
\]

\textbf{Green-St. Venant strain tensor}:
\[
\E 	= \frac{1}{2}(\F^T\F - \I)
	= \frac{1}{2}(\C - \I)
	= \frac{1}{2}(\U^2 - \I)\tag{7.6,7.7,7.8}
\]
As rotations tensors are orthogonal, \E vanishes when \F is a rotation.
We now have properties
\begin{enumerate}
	\item (M3) \U,\C and \E map material vectors to material vectors
	\item (M4) \V and \B map spatial vectors to spatial vectors
	\item (M5) \R maps material vectors to spatial vectors
\end{enumerate}

\subsection{Fibers. Properties of the Tensors U and C}



\end{document}