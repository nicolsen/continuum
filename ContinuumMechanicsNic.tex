\documentclass{article}
\usepackage[top=1.5in,bottom=1in,left=0.5in,right=0.5in]{geometry}
\newcommand{\Chi}{\boldsymbol{\chi}}

\usepackage{amsmath}
\usepackage{amssymb}
\usepackage{bm}

\newtheorem{theorem}{Theorem}[section]
\newtheorem{definition}{Definition}[section]
\newtheorem{remark}{Remark}[section]


\renewcommand{\a}{\textbf{a}}
\renewcommand{\b}{\textbf{b}}
\renewcommand{\c}{\textbf{c}}
\renewcommand{\d}{\textbf{d}}
\newcommand{\e}{\textbf{e}}
\newcommand{\f}{\textbf{f}}
\newcommand{\g}{\textbf{g}}
\newcommand{\h}{\textbf{h}}
\renewcommand{\i}{\textbf{i}}
\renewcommand{\j}{\textbf{j}}
\renewcommand{\k}{\textbf{k}}
\renewcommand{\l}{\textbf{l}}
\newcommand{\m}{\textbf{m}}
\newcommand{\n}{\textbf{n}}
\renewcommand{\o}{\textbf{o}}
\newcommand{\p}{\textbf{p}}
\newcommand{\q}{\textbf{q}}
\renewcommand{\r}{\textbf{r}}
\newcommand{\s}{\textbf{s}}
\renewcommand{\t}{\textbf{t}}
\renewcommand{\u}{\textbf{u}}
\renewcommand{\v}{\textbf{v}}
\newcommand{\w}{\textbf{w}}
\newcommand{\x}{\textbf{x}}
\newcommand{\y}{\textbf{y}}
\newcommand{\z}{\textbf{z}}
\newcommand{\A}{\textbf{A}}
\newcommand{\B}{\textbf{B}}
\newcommand{\C}{\textbf{C}}
\newcommand{\D}{\textbf{D}}
\newcommand{\E}{\textbf{E}}
\newcommand{\F}{\textbf{F}}
\newcommand{\G}{\textbf{G}}
\renewcommand{\H}{\textbf{H}}
\newcommand{\I}{\textbf{I}}
\newcommand{\J}{\textbf{J}}
\newcommand{\K}{\textbf{K}}
\renewcommand{\L}{\textbf{L}}
\newcommand{\M}{\textbf{M}}
\newcommand{\N}{\textbf{N}}
\renewcommand{\O}{\textbf{O}}
\renewcommand{\P}{\textbf{P}}
\newcommand{\Q}{\textbf{Q}}
\newcommand{\R}{\textbf{R}}
\renewcommand{\S}{\textbf{S}}
\newcommand{\T}{\textbf{T}}
\newcommand{\U}{\textbf{U}}
\newcommand{\V}{\textbf{V}}
\newcommand{\W}{\textbf{W}}
\newcommand{\X}{\textbf{X}}
\newcommand{\Y}{\textbf{Y}}
\newcommand{\Z}{\textbf{Z}}


\newcommand{\del}{\nabla}
\newcommand{\eqdef}{:=}
\newcommand{\curly}[1]{\mathcal{#1}}


\begin{document}

\begin{titlepage}
	\title{Notes on:\\The Mechanics and Thermodynamics of Continua}
	\date{\today}
	\maketitle
\end{titlepage}

\tableofcontents
\newpage	
	
\section{Motion of a Body}
$B$ is a reference body containing points $\X$ which are material points.
There is a one-to-one function
$$\x = \Chi(\X,t)$$
taking reference material points $\X$ to spatial points $\x$ at time $t$.
We require $$J(\X,t) \eqdef \det \del \Chi_t(\X) > 0$$
where J is the volumetric Jacobian of the mapping $\Chi_t$ at $\X$.
Region occupied by body $B$ at time $t$ is
$$\curly{B}_t = \Chi_t(B)$$
is the deformed body at time $t$.

\subsection{Convection of Sets with the Body} $A$ is a material set. Then \textbf{deforms to} $\curly{A}_t$ at time $t$. 
$\curly{A}_t$ \textbf{convects with the body} if there is a set $A$ of material points such that
$$\curly{A}_t = \Chi_t(A)$$ for all $t$.
Note that material cannot cross the boundary of a spatial set which convects with the body. Also note that if $\X$ is on $\partial B$ (boundary), then $\Chi(\X,t)$ is on $\partial \curly{B}_t$ for all time $t$ and conversely.

\section{The Deformation Gradient}
The \textbf{Deformation gradient} of a body is $$\textbf{F} = \del \Chi, \hspace{1cm} F_{ij} = \frac{\partial \chi_i}{\partial X_j},$$ the Jacobian matrix of $\x = \x(\X)$.
As above
$$J = \det \textbf{F} > 0.$$
\subsection{Approximation of a Deformation by a Homogeneous Deformation}
\subsubsection{Homogeneous Deformations} Fix time $t$ so that $$\Chi(\X) \equiv \Chi_t(\X).$$
$\Chi$ is a \textbf{homogeneous deformation} if $\textbf{F}(\X) \equiv \textbf{F}(\X,t)$ is independent of $\X$. So
$$\Chi(\X) - \Chi(\Y) = \F(\X - \Y)$$
for \textbf{all} material points $\X$ and $\Y$. By the above, \underline{$\F$ maps material vectors to spatial vectors.} Then, also, $\X - \Y = \F^{-1}[\Chi(\X) - \Chi(\Y)]$ so that $\F^{-1}$ maps spatial vectors too material vectors. Taking the inner product with a spatial vector $\s$ gives
$$\s \cdot [\Chi(\X) -\Chi(\Y)] = \s \cdot [\F(\X - \Y)]  = (\F^T\s)\cdot(\X - \Y)$$ so that $\F^T$ maps spatial vectors to material vectors.\\
Summarizing the mapping properties:
\begin{enumerate}
	\item $\F$ and $\F^{-T}$ map material vectors to spatial vectors
	\item $\F^{-1}$ and $\F^T$ map spatial vectors to material vectors
\end{enumerate}

\subsubsection{General Deformations}
Let $\Chi_t$ be an arbitrary deformation. Taylor expanding the deformation about material point $\X$ gives 
$$\underline{\Chi_t(\Y) - \Chi(\X) = \F(\X,t)(\Y - \X)} + o(|\Y - \X) \hspace{1cm} \text{as } |\Y - \X| \to 0.$$ Therefore, $\F(\X,t)(\Y - \X)$ is an approximation of $\Chi_t(\Y) - \Chi(\X)$. Also, since $\X$ is fixed in the Taylor expansion, $\F(\X,t)$ is constant. Thus
the underlined portion is the second definition of a homogeneous deformation. Therefore, \emph{within a neighborhood of a material point $\X$ and to within an error of $o(|\Y -\X|)$, a deformation behaves like a homogeneous deformation.}\\
So with $o(|\Y - \X|)$ small, we have:
\begin{enumerate}
	\item (M1) $F(\X,t)$ can be thought of as a mapping of an infinitesimal neighborhood of $\X$ in the reference body to an infinitesimal neighborhood of $\x - \Chi_t(\X)$ in the deformed body.
	\item (M2) This gives an asymptotic meaning to the formal relation
	$$\textbf{dx} = \F(\X,t)\textbf{dX}$$
\end{enumerate}
Now, we have that the mapping properties for a homogeneous deformation hold pointwise for the deformation gradient in an arbitrary deformation. For example, for a given $\X$, the linear transformation $\F(\X,t)$ associates with each material vector $\m$ a spatial vector $\s = \F(\X,t)\m$.

\subsection{Convection of Geometric Quantities} 
\subsubsection{Infinitesimal Fibers}Define the temporally constant material vector field $\textbf{f}_R$ associated with a given spatial vector file $\textbf{f}$ by
\[
\textbf{f}(\x,t) = \F(\X,t)\textbf{f}_R(\X) \hspace{1cm} \x = \Chi_t(\X) \tag{6.8} \label{eq:6.8}
\]
 for all $\X$ and $t$.

Now by the above statements about the local homogeneity of deformation, we can see 
equation~\ref{eq:6.8} becomes
\[
\epsilon\textbf{f}(\x,t) = \F(\X,t)(\epsilon\textbf{f}_R(\X)) \tag{6.9} \label{eq:6.9}
\] 
for $\epsilon > 0$. This can be considered as describing the local deformation when the neighborhood of $\X$ under consideration is magnified by a factor of $\epsilon^{-1}$.\\

In equation~\ref{eq:6.8}, $\f_R(\X)$ is an \textbf{infinitesimal undeformed fiber} and $\f(\x,t) = \F(\X, t)\f_R(\X)$ is the corresponding \textbf{(infinitesimal) deformed fiber}. We can see the deformed fiber as \textbf{embedded} in the deforming body $\curly{B}_t$ and we say $\f(\x,t)$ convects with the body.\\

\textbf{f convects with the body} and \textbf{f is convecting} mean that there is a fixed (time independent) material vector field $\f_R$(\X) such that equation~\ref{eq:6.8} holds.\\

\subsubsection{Curves}
$C$ is a \textbf{material curve} with parameterization $\hat{\X}(\lambda)$, $\lambda\in [\lambda_0,\lambda_1]$ which does not intersect itself. The corresponding \textbf{spatial curve} is $\curly{C}_t = \Chi_t(C)$. Note the time-dependent parameterization. Then $\curly{C}_t$ is a curve \textbf{embedded} in the deforming body.

\subsubsection{Tangent Vectors}
Given $\X$ on $C$, the tangent to $C$ at $\X$ is
\[
\boldsymbol{\tau}_R(\X) = \frac{d\hat{\X}(\lambda)}{d\lambda} \tag{6.12} \label{eq:6.12}
\]
Then the corresponding tangent to $\curly{C}_t$ at $\x$ is
\[
\boldsymbol{\tau}(\x,t)= \frac{\partial \hat{\x}_t(\lambda)}{\partial\lambda} \tag{6.13}\label{eq:6.13}
\]
which gives
\[
\boldsymbol{\tau}(\x,t) = \F(\X,t)\boldsymbol{\tau}_R(\X) \tag{6.15}\label{eq:6.15}
\]


\begin{theorem}[Transformation Law for Tangent Vectors]
	At each time, the relation (\ref{eq:6.15}) associates with any vector $\boldsymbol{\tau}_R$ at $\X$ a vector $\boldsymbol{\tau}$ at $\x = \Chi_t(\X)$ with the following property: if $\boldsymbol{\tau}_R$ is tangent to a material curve at $\X$, then $\boldsymbol{\tau}$ is tangent to the corresponding deformed curve through $\x$.
\end{theorem}

\subsubsection{Bases}
Fix a material basis
$$\lbrace \textbf{m}_i(\X)\rbrace = \lbrace \textbf{m}_1(\X),\textbf{m}_2(\X),\textbf{m}_3(\X) \rbrace.$$
Then the associated spatial basis is
\[
\lbrace \textbf{s}_i(\x,t)\rbrace = \lbrace \F(\X,t)\textbf{m}_i(\X)\rbrace \tag{6.16}\label{eq:6.16}
\]

at $\x = \Chi(\X, t)$.
This spatial basis convects with the body, ie is embedded in the deforming body.

\section{Stretch, Strain, and Rotation}
\subsection{Stretch and Rotation Tensors. Strain}
The polar decomposition (rotation $\textbf{R}$ and positive-definite symmetric tensors \textbf{U} and \textbf{V}) of the deformation gradient:
\[
\F = \textbf{R}\textbf{U} = \textbf{V}\textbf{R} \tag{7.1}\label{eq:7.10}
\]

\textbf{U} is the \textbf{right stretch tensor} and \textbf{V} is the \textbf{right stretch tensor}.

The following is good for theoretical but difficult in application
\[
\centering
\textbf{U} = \sqrt{\F^T\F} \hspace{2cm} \text{and} \hspace{2cm}\textbf{V} = \sqrt{\F\F^T} \tag{7.2} \label{eq:7.2}
\]

\textbf{Left and Right Cauchy-Green (deformation) tensors C and B}:
\[
\begin{array}{lr}
\textbf{C} = \textbf{U}^2 = \F^T\F,  &  C_{ij} = F_{ki}F_{kj} = \frac{\partial \Chi_k}{\partial\X_i}\frac{\partial\Chi_k}{\partial\X_j}\\

\textbf{B} = \textbf{V}^2 = \F\F^T,  &  B_{ij} = F_{ik}F_{jk} = \frac{\partial \Chi_i}{\partial\X_k}\frac{\partial\Chi_j}{\partial\X_k}
\end{array} \tag{7.3}\label{eq:7.3}
\]

Then,
\[
\textbf{V} = \textbf{RUR}^T \hspace{1cm} and \hspace{1cm} \textbf{B} = \textbf{RCR}^T \tag{7.4} \label{eq:7.4}
\]

and
\[
\text{\U, \V, \C, and \B\space are symmetric and positive-definite.}\tag{7.5}\label{eq:7.5}
\]

\textbf{Green - St. Venant strain tensor}:
\[
\E 	= \frac{1}{2}(\F^T\F - \I)
	= \frac{1}{2}(\C - \I)
	= \frac{1}{2}(\U^2 - \I)\tag{7.6,7.7,7.8}
\]
As rotation tensors are orthogonal, $\E$ vanishes when $\F$ is a rotation.
We now have properties
\begin{enumerate}
	\item (M3) $\U$, $\C$, and $\E$ map material vectors to material vectors
	\item (M4) $\V$ and $\B$ map spatial vectors to spatial vectors
	\item (M5) $\R$ maps material vectors to spatial vectors
\end{enumerate}

\subsection{Fibers. Properties of the Tensors U and C}
\subsubsection{Infinitesimal Fibers}
Infinitesimal undeformed fibers and $\f_R$ and $\bar{\f}_R$ and corresponding deformed fibers
\[
	\f = \F\f_R \hspace{1cm} and \hspace{1cm} \bar{\f} = \F\bar{\f}_R\tag{7.9}\label{eq:7.9}
\]

Then
\[
	\f\cdot\bar{\f} = (\R\U\f_R)\cdot(\R\U\bar{\f}_R) = \U\f_R\cdot\U\bar{\f}_R = \f_R\cdot\U^2\bar{\f}_R
	 = \f_R\cdot\C\bar{\f}_R\tag{7.11}\label{eq:7.11}
\]

So,
\[
	|\f| = |\U\f_R|\tag{7.12}\label{eq:7.12}
\]

So applying the right stretch tensor gives the deformed length of an infinitesimal fiber.\\

Define: $\theta = \angle(\f_R,\bar{\f_R})$ the angle between fibers.

Then,
by \ref{eq:7.10} and \ref{eq:7.12},

\[
	\frac{\f\cdot\bar{\f}}{|\f||\bar{\f}} = \frac{\C\f_R\cdot\U\bar{\f}_R}{|\U\f_R||\U\bar{\f}_R|}
\]

So,

\[
	\angle(\f,\bar{\f}) = \angle(\U\f_R,\U\bar{\f}_R) \tag{7.13} \label{eq:13}
\]

so applying the right stretch tensor gives the angle between infinitesimal deformed fibers.

\subsubsection{Finite Fibers}
Consider material and spatial line segments
\[
	\Delta\X = \Y - \X \hspace{1cm} and \hspace{1cm} \Delta\x = \Chi(\Y) - \Chi(\X)
\]
with $\Delta\X > 0.$

Then we know 
\[
	\Delta\x = \F(\X)\Delta\X + o(|\Delta\X|) \hspace{1cm} as \hspace{1cm} |\Delta\X| \to 0. \tag{7.14}\label{eq:7.14}
\]

Useful to think of $\Delta\X$ as of an undeformed fiber of finite length $L$ and direction $\e$ at $\X$, so
\[
	\Delta\X = L\e, \qquad |\e| = 1\tag{7.15}\label{eq:7.15}
\]

So the corresponding deformed fiber is

\[
	\Delta\x = L\F(\X)\e + o(L) \qquad as \qquad L\to 0
\]

Then the following limit gives the \textbf{stretch vector}
\[
	\lim\limits_{L\to 0}\frac{\Delta\x}{L} = \F(\X)\e\tag{7.16}\label{eq:7.16}
\]

This is called the stretch vector because it is the limiting value of the deformed fiber measured per \textbf{unit length} of the undeformed fiber in direction $\e$.

Thus the \textbf{stretch} is
\[
	\lambda = \lim\limits_{L\to 0}\frac{|\Delta\x|}{L} = |\F(\X)\e|\tag{7.17}\label{eq:17}
\]

From earlier, taking $\f_R=\e$ gives
\[
	\lambda = |\U(\X)\e| \qquad \lambda^2 = \e\cdot\C(\X)\e\tag{7.18}\label{eq:7.18}
\]
\begin{remark}
	The right stretch tensor $\U$ determines the stretch $\lambda$ at $\X$ relative to any material direction $\e$ by $\lambda = |\U(\X)\e|$.
\end{remark}

Now, taking two fibers from $\X$ of the same length $L$ with directions $\e_1$ and $\e_2$.
Then
\[
	\lim\limits_{L\to 0} \left( \frac{(\Delta\x)_1)}{L} \cdot \frac{(\Delta\x)_2}{L} \right)
	= \lim\limits_{L\to 0} \left( \frac{(\Delta\x)_1)}{L} \right) \cdot \lim\limits_{L\to 0} \left( \frac{(\Delta\x)_2}{L} \right)
	= \U(\X)\e_1\cdot\U(\X)\e_2
	= \e_1\cdot\C(\X)\e_2 \tag{7.21-7.22}\label{eq:7.21}
\]

\begin{remark}
	The right Cauchy-Green tensor $\C(\X)$ characterizes inner products of stretch vectors at $\x$
\end{remark}

Let $\theta_L$ be the angle between the deformed fibers $(\Delta\x)_1$ and $(\Delta\x)_2$, as before.
Then,
\[
	\theta_L = \angle((\Delta\x)_1,(\Delta\x)_2) = \cos^{-1}\left( \frac{(\Delta\x)_1 \cdot (\Delta\x)_2}{|(\Delta\x)_1||(\Delta\x)_2|} \right)
\]
After math cheese,
\[
	\lim\limits_{L\to 0}\theta_L = \angle(\U(\X)\e_1, \U(\X)\e_2). \tag{7.23}\label{eq:7.23}
\]

\begin{remark}
	Let $(\Delta\x)_1$ and $(\Delta\x)_2$ be the deformed fibers corresponding to fibers at $\X$ of finite length $L$ in directions $\e_1$ and $\e_2$. Then, as $L\to 0$, the angle between
	$$\frac{(\Delta\x_1)}{L} \; \text{and} \; \frac{(\Delta\x)_2}{L}$$
	tends to the angle between $\U(\X)\e_1$ and $\U(\X)\e_2$.
\end{remark}

\subsection{Principle Stretches and Principal Directions}
As $\U$ and $\V$ are symmetric and positive-definite, the have spectral representations of the form
\[
	\U = \lambda_i\r_i \otimes\r_i = \sum_{i=1}^{3}\lambda_i\r_i\otimes\r_i \qquad
	\V =  \lambda_i\l_i\otimes\l_i
\]
where
\begin{enumerate}
	\item $\lambda_i > 0 \, \forall i$, the \textbf{principal stretches} are eigenvalues of $\U$ and also of $\V$.
	\item $\r_1$, $\r_2$, and $\r_3$ are the \textbf{right principal directions} and eigenvectors of $\U$
	$$\U\r_i = \lambda_i\r_i \qquad \text{(no sum on i)}$$
	\item $\l_1$, $\l_2$, and $\l_3$ are the \textbf{left principal directions} and the eigenvectors of $\V$
	$$\V\l_i = \lambda_i\l_i \qquad \text{(no sum on i)}$$
\end{enumerate}

Thus, $\V = \R\U\R^T$ gives
\[
	\sum_{i=1}^{3}\lambda_i\R\r_i\otimes\R\r_i = \sum_{i=1}^{3}\lambda\l_i\otimes\l_i
\]
Therefore,
\[
	\l_i = \R\r_i, \qquad i =1,2,3\tag{7.27}\label{eq:7.27}
\]

We then have the following expressions using the principal stretches and directions:
\[
	\left.
	\begin{array}{l}
	\C = \lambda_i^2\r_i\otimes\r_i\\
	\B = \lambda_i^2\l_i\otimes\l_i\\
	\E = \frac{1}{2}(\lambda_i^2 - \I)\r_i\otimes\r_i
	\end{array}
	\right\rbrace\tag{7.28}\label{eq:7.28}
\]
where each is summed over $i$.

Also,
\[
	\F = \R\U = \lambda_i\l_i\otimes\r_i 
\]

Also the logarithmic strain tensors of Hencky are scary.

\section{Deformation of Volume and Area}
Time is fixed throughout this section.
\subsection{Deformation and Normals}
Given: $\X$, $\x = \Chi(\X)$, material surface $S$ containing $\X$, spatial surface $\curly{S} = \Chi(S)$ containing $\x$, and nonzero material normal $\n_R$ with
\[
\n_R\cdot\t_R = 0 \; \forall \, \t_R\text{ tangent to S at } \X .\tag{8.1}\label{eq:8.1}
\]
Then after some work, the corresponding spatial normal to $\curly{S}$ at $\x$ is
\[
\n = \F^{-T}(\X)\n_R\tag{8.4}\label{eq:8.4}
\]

\subsection{Deformation of Volume}
Consider basis $\lbrace \F(\X)\e_i\rbrace$, then the volume spanned by these vectors is
\[
(\F\e_1\times\F\e_2)\cdot\F\e_3 = \det \F = J > 0 \tag{8.5}\label{eq:8.5}
\]

Now, define the deformed fiber at $\x = \Chi(\X)$ deformed from fiber of length $l$ in direction $\e_i$ at $\X_i$ by
\[
\boldsymbol{\delta}_i(l) = \Chi(\X + l\e_i) - \Chi(\X)\tag{8.6}\label{eq:8.6}
\]


Now, define the volumes spanned by undeformed and deformed fibers
\[
\Delta\nu_R(l) \eqdef l^3 (\e_1\times\e_2)\cdot\e_3 = l^3 \qquad and
\qquad \Delta\nu(l) \eqdef (\boldsymbol{\delta}_1(l)\times\boldsymbol{\delta}_2(l))\cdot\boldsymbol{\delta}_3(l)
\]

Taylor expanding gives
\[
\boldsymbol{\delta}_i(l) = l\F\e_i + o(l) \tag{8.8}\label{eq:8.8}
\]

so that:
\begin{remark}
	For small undeformed fibers making up a cube of volume $\Delta\nu_R$, the corresponding deformed fiber parallelepiped has volume approximately $J\Delta\nu_R = \det \F\Delta\nu_R$.
\end{remark}
\subsection{Deformation of Area}

Let plane $\Pi_R$ with normal $\n_R$ at material point $\X$ with $|\n_R| = 1$, choose basis $\lbrace \e_i \rbrace$ such that $\e_3 = \n_R$.

Then the volume spanned by $\F\e_1$, $\F\e_2$, and $\F\n_R$ is
\[
	(\F\e_1\times\F\e_2)\cdot\F\n_R = J \tag{8.11}\label{eq:8.11}
\]

Now, the area spanned by $\F(\X)\e_1$ and $\F(\X)\e_2$ at $\x$ is
\[
	j(\X)\eqdef |\F(\X)\e_1\times\F(\X)\e_2|\tag{8.13}\label{eq:8.13}	 
\]

where $j$ is the \textbf{Areal Jacobian}.\\

Thus,
\[
	j = J|\F^{-T}\n_R|\tag{8.14}\label{eq:8.14}
\]

relates the areal and volumetric Jacobians. Note that $j$ depends on the choice of normal $\n_R$.
Then, since the deformed normal is $\n = \F^{-T}\n_R$, we have
\[
	j?\frac{\n}{|\n|}=\F^C\n_R\tag{8.17}\label{eq:8.17}
\]

where the \textbf{cofactor of F} is
\[
	\F^C = (\det\F)\F^{-T}\tag{8.18}\label{eq:8.18}
\]

Now, similar to before, we have
\[
	\Delta a_R(l) \eqdef l^2|\e_1\times\e_2| = l^2 \qquad and \qquad \Delta a(l) \eqdef |\boldsymbol{\delta}_1(l)\times\boldsymbol{\delta}_2(l)| \tag{8.19} 
\]

Then,
\[
	\Delta a = j\Delta a_R + o(\Delta a_R) \qquad as \qquad \Delta a_R \to 0\tag{8.20}
\]
\begin{remark}
	Given and undeformed fiber square of area $\Delta a_R$, the corresponding deformed fiber parallelogram has area $j\Delta a_R$ to an error of $o(\Delta a_R)$.
\end{remark}
Finally,
\[
	\Delta a\frac{\n}{|\n|} = \Delta a_R\F^C\n_R + o(\Delta a_R)\tag{8.21}
\]


\section{Material and Spatial Descriptions of Fields}
Note that for a fixed $t$, $\Chi$ is invertible. We call $\X = \Chi^{-1}(\x,t)$ the \textbf{reference map}.\\

We now define the \textbf{velocity} $\dot{\Chi}(\X,t)$ as a function of $\v(\x,t)$
\[
	\v(\x,t)=\dot{\Chi}(\Chi^{-1}(\x,t),t) \qquad \text{or equivalently}\qquad \dot{\Chi}(\X,t) = \v(\Chi(\X,t),t)  \tag{9.1}
\]
where $\v$ is the \textbf{spatial description of the velocity}: $\v(\x,t)$ is the velocity which, at a fixed time $t$, is the velocity of the material point that occupies the spatial point $\x$, ie the \textbf{spatial} velocity of the point.

\begin{remark}
	For a function $\varphi$ on the body, there is a \textbf{material description} $\varphi (\X ,t)$ and a \textbf{spatial description} $\phi(\x ,t)$ with the relations
	$$\phi(\x ,t) = \varphi(\Chi^{-1}(\x ,t),t)$$
	and
	$$\varphi(\X ,t) = \phi(\Chi(\X ,t),t)$$
\end{remark}
\subsection{Gradient, Divergence and Curl}
$$\del,\; \Divergence,\; and \; \Curl$$
denote the \textbf{material} gradients, divergence, and curl. So,
$$ (\del \h)_{ij} = \pd{h_i}{X_j}, \qquad \Divergence\h = \pd{h_i}{X_i}, \qquad (\Curl\h)_{ij} = \epsilon_{ijk}\pd{h_k}{X_j}$$
	
Similarly,
$$\grad, \, \divergence, \, \curl$$
denote \textbf{spatial} gradients, divergence, and curl:
$$(\grad\g)_{ij} = \pd{g_i}{x_j}, \qquad \divergence\g = \pd{g_i}{x_i}, \qquad (\curl\g)_i = \epsilon
_{ijk}\pd{g_k}{x_j}$$

\subsection{Material and Spatial Time Derivatives}

\[
	\dot{\varphi}(\X,t) = \pd{\varphi(\X,t)}{t}\tag{9.3}
\]
is the material time derivative, holding $\X$ fixed.

\[
	\varphi^\prime(\x,t) = \pd{\varphi(\x,t)}{t}\tag{9.4}
\]
is the spatial time derivative, holding $\x$ fixed.

The relationship between them is
\[
	\dot{\varphi}(\x,t) =\left.\left[ \pd{}{t}\right|_{\X}\varphi(\Chi(\X,t),t) \right]_{\X=\Chi^{-1}(\x,t)}\tag{9.5}
\]

\begin{remark}[Time-Derivative Identities]

\[
	\dot{\varphi} = \varphi ^\prime + \v\cdot \grad\varphi
\]
\[
	\dot{\g} = \g^\prime + (\grad\g)\v
\]
\end{remark}

In particular, the acceleration is given by
\[
	\dot{\v} = \v^\prime + (\grad\v)\v
\]

\subsection{Velocity Gradient}

The spatial tensor field
\[
	\L = \grad\v\tag{9.10}
\]
is the \textbf{velocity gradient}.

Therefore,
\[
	\dot{\F} = \L\F\tag{9.11}
\]
Note that $\L$ maps spatial vectors to spatial vectors.

We then have several results:


\[
	\tr\L = \divergence\v\tag{9.14}
\]

\[
	\dot{J} = \dot{\det\J} = J\tr\L\tag{9.15}
\]

\[
	\dot{J} = J\divergence\v\tag{9.16}
\]

Note that this relates to the transport of volume.

\[
	\dot{\F}^T = \F^T\L^T\tag{9.17}
\]

\[
	\dotbar{\F^{-1}} = -\F^{-1}\dot{\F}\F^{-1}\tag{9.18}
\]

\[
	\dotbar{\F\inv} = -\F\inv\L\tag{9.19}
\]

\[
	\dotbar{\F^{-T}} = -\L^T\F^{-T}\tag{9.20}
\]


\subsection{Commutator Identities}
\[
	\grad\dot{\varphi} = \dotbar{\grad\varphi} + \L^T\grad\varphi \qquad \pd{\dot{\varphi}}{x_i} - \dotbar{\pd{\varphi}{x_i}} + L_{ji}\pd{\varphi}{x_j}\tag{9.23}
\]

\[
	\grad\dot{\g} = \dotbar{\grad\g} + (\grad\g)\L \qquad \pd{\dot{g_j}}{x_i} + \pd{g_j}{x_k}L_{ki}\tag{9.24}
\]

These give that unlike the \textbf{spatial} time derivative and th \textbf{spatial} gradient, the \textbf{material} gradient and the \textbf{spatial} gradient do not generally commute.

\subsection{Particle Paths}
Given material point $\X$, the \textbf{particle path} of $\X$ is the spatial trajectory of the particle $\X$ over time:

\[
	\p(t) = \Chi(\X, t)\tag{9.27}
\]
for the fixed $\X$.


Then,
\[
	\dot{\p}(t) = \v(\p(t),t)\tag{9.28}
\]
\subsection{Stetching of Deformed Fibers}
Define the distance between particle paths of material points:
\[
	\delta(t) = |\Chi(\X_1,t) - \Chi(\X_2,t)|\tag{9.29}
\]

\begin{remark}
	$\delta(t)$ gives the length of the deformed fiber corresponding to the undeformed fiber:
	$$\Delta\X = \X_2 - \X_1$$
	at time $t$. Then, $\dot{\delta}(t)$ is the rate at which the deformed fiber is being stretched.
\end{remark}

Then we have
\[
	\delta(t)\dot{\delta}(t) = (\x_1-\x_2)\cdot[\v(\x_1,t) - \v)(\x_2,t)]\tag{9.30}
\]
where $\x_1$ and $\x_2$ are the corresponding spatial points.

\section{Special Motions}
\subsection{Rigid Motions}
A motion $\Chi$ is \textbf{rigid} if, at each time $t$,
$$\pd{}{t}|\Chi(\X,t) - \Chi(\Y,t)| = 0$$
for all material points $\X$ and $\Y$, or equivalently:
$$(\x - \y)\cdot[\v(\x,t) - \v(\y,t)] = 0$$
for all $\x$ and $\y$ in $\curly{B}_t$.
ie, a motion is rigid if every pair of spatial points stays at the same distance throughout the motion.

Now, define \textbf{spin} by
\[
	\W(t) = \grad\v(\x,t)
\]
so that
\[
	\v(\x,t) = \v(\y,t) + \W(t)(\x-\y).
\]

Thus, for a rigid motion, the velocity gradient is just the spin:
\[
	\L = \W\tag{10.1}
\]

Now let $\w(t)$ be the axial vector of $\W(t)$ so that
\[
	\W = \w\times
\]

and
\[
	\v(\x,t) = \v(\y,t) + \w(t)\times(\x-\y)\tag{10.2}
\]
by definition.
In this case, $\w(t)$ is the \textbf{angular velocity} of the motion.

Now the curl of $\v$ is 
\[
	\curl\v = 2\w\tag{10.3}
\]
for rigid motion.

Note that for $\w \ne \textbf{0}$ we have:
$$\forall\a, \, \w\times\a=\textbf{0} \iff \W\a=\textbf{0}$$
We therefore call the \textbf{spin axis} the subspace $\curly{L}$ of vectors $\a$ such that

\[
	\W\a = \textbf{0}\tag{10.4}
\]

The \textbf{rigid velocity field} is a spatial field of the form $\v(\x,t) = \v(\y,t) + \boldsymbol{\lambda}(t)\times(\x - \y)$ or equivalently

\[
	\v(\x,t) = \boldsymbol{\alpha}(t) + \boldsymbol{\lambda}(t)\times(\x-\o)\tag{10.5}
\] 
\subsection{Motions Whose Velocity Gradient is Symmetric and Spatially Constant}
In this case, write $\D = \L$ for the symmetric, spatially constant velocity gradient.
\[
	\v(\x) = \D(\x-\y)
\]
for some $\y$. Since $\D$ is symmetric, we have the spectral decomposition:
\[
	\D = \sum_{i=1}^{3}\alpha_i\m_i\otimes\m_i
\]
with the three "axes" being orthogonal. Thus, we need only consider
\[
	\v(\x) = \alpha(\e\otimes\e)(\x-\y)\tag{10.6}
\]

Fix the orthonormal basis $\lbrace \e_i\rbrace$ and suppose $\e = \e_i$. 

Then, $\v$ has only a one nontrivial component: $v_i = \v\cdot\e = \v\cdot\e_i$ with
\[
	v_i(\x) = \alpha(x_i-y_i)
\]

\begin{remark}
	Thus, all together, this means that, up to an additive constant, every velocity field with symmetric gradient and constant is the sum of three velocity fields of the for (10.6) with "axes" mutually orthogonal.
\end{remark}

\section{Stretching and Spin in an Arbitrary Motion}
\subsection{Stretching and Spin as Tensor Fields}
Let $\v$ be a general velocity field. Taylor expanding about arbitrary spatial point $\y$ give
\[
	\v(\x) - \v(\y) = \L(\y)(\x - \y) + o(|\x - \y|)\tag{11.1}
\]
as $\x\to\y$.

Define the symmetric and skew parts of $\L$ as
\[
\begin{array}{l}
\D = \frac{1}{2}(\L + \L^T)\\
\W = \frac{1}{2}(\L - \L^T)\tag{11.2}
\end{array}
\]
So,
\[
	\L = \D + \W\tag{11.3}
\]
Thus, the Taylor expansion becomes
\[
	\v(\x) - \v(\y) = \W(\y)(\x - \y) + \D(\y)(\x - \y) + o(|\x - \y|)
\]
We then call $\W$ \textbf{spin} and $\D$ \textbf{dilation} (book calls stretching) and the \textbf{spin axis} at $(\y,t)$ is the subspace $\curly{L}$ of $\a$ such that
\[
	\W(\y,t)\a = \boldsymbol{0}\tag{11.4}
\]
\begin{remark}
	$\D$ and $\W$ map spatial vectors to spatial vectors.
\end{remark}

Relating these in with left and right stretch tensors gives:
\[
	\L = \dot{\R}\R^T + \R\dot{\U}\U^{-1}\R^T\tag{11.6}
\]

We know that $\dot{\R}\R^T$ is skew, therefore we have
\[
\begin{array}{l}
	\D = \R[\sym(\dot{\U}\U\inv)]\R^T\\
	\W = \dot{\R}\R^T + \R[\skw(\dot{\U}\U\inv)]\R^T\tag{11.8}
\end{array}
\]
$\W$ is the sum of rotational spin and stretch spin:
\[
\begin{array}{l}
	\W = \W_{rot} + \W_{str}\\
	\W_{rot} = \dot{\R}\R^T \qquad \W_{str} = \R[\skw(\dot{\U}\U\inv)]\R^T
\end{array}
\]

Now,
\[
	\F^T\D\F = \dot{\E}\tag{11.10}
\]
relates the dilation $\D$ and the time-derivative of the Green - St. Venant strain $\E$.

\subsection{Properties of D}
Define a unit vector$\e$. Define $\delta_l(\tau)$ as the distance at time $\tau$ between the particle paths that pass through the points $\x$ and $\x + l\e$ at a reference time $\t$. Another way: $\x$, $\x +l\e$, and time $t$ specify two particle paths. Then $\delta_l(\tau)$ is the distance between those paths at some time $\tau$.
Then,
\[
	\lim\limits_{l\to 0}\frac{\dot{\delta}_l(t)}{\delta_l(t)} = \e\cdot\D(\x,t)\e\tag{11.12}
\]
\begin{remark}
	$\D$ characterizes the rate at which deforming fibers are stretched
\end{remark}

Now consider angles:
Define $\X$, $\Y_1$, and $\Y_2$ as material points occupying spatial points $\x$, $\y_1= \x + l\e_1$, and $\y_2 = \x + l\e_2$ at time $t$. Then, let the angle $\theta_l(\tau)$ between deformed fibers
\[
	\u_\alpha(\tau) = \Chi (\Y_\alpha, \tau ) - \Chi (\X, \tau ), \qquad \alpha = 1,2
\]
ie
\[
	\theta_l(\tau) = \angle(\Chi(\Y_1, \tau), \Chi(\X, \tau), \Chi(\Y_2, \tau))
\]
at time $t$.

After lots of angle math:
\begin{remark}
	$\D$ characterizes rates at which angles between deforming fibers change
\end{remark}

\[
	\lim\limits_{l\to 0}\dot{\theta}_l(t) = -2\e_1\cdot\D(\x,t)\e_2
\]

\subsection{Stretching and Spin Using the Current Configuration as Reference}
We can express the motion in terms of the spatial point $\x$ and the time $\tau$:
\[
	\boldsymbol{\xi} \eqdef \Chi(\Chi\inv(\x,t),\tau) = \Chi_{(t)}(\x,\tau)\tag{11.13}
\]
$\bs{\xi}$ is the spatial point occupied at time $\tau$ by the material point that occupies the spatial point $\X$ at time $\t$.
The \textbf{relative deformation gradient} is the
\[
	\F_{(t)}(\x,\tau) = \del\Chi_{(t)}(\x,\tau)\tag{11.14}
\]
Note
\[
	\Chi(\X,\tau) = \Chi_{(t)}(\Chi(\X,t),\tau)\tag{11.15}
\]

Then we have
\[
	\F_{\tau} = \F_{(t)}(\tau)\F(t)\tag{11.16}
\]
so,
\[
	\F_{(t)}(t) = \I\tag{11.17}
\]

Then, as with the first deformation gradient, we may take a polar decomposition:
\[
	\F_{(t)} = \R_{(t)}\U_{(t)} = \V_{(t)}\R_{(t)}\tag{11.18}
\]
As expected, $\R_{(t)}$ is the \textbf{relative rotation}, $\U_{(t)}$ and $\V_{(t)}$ are the \textbf{relative stretch tensors}, and the \textbf{relative Cauchy-Green tensors} are $\C_{(t)} = \U_{(t)}^2$ and $\B_{(t)} = \V_{(t)}^2$.

Thus,
\[
	\R_{(t)}(t) = \U_{(t)}(t) = \V_{(t)}(t) = \C_{(t)}(t) = \B_{(t)}(t) = \I\tag{11.19}
\]

Then for the velocity gradient we have:
\[
	\L(\x,t) = \left.\pd{\F_{(t)}(\x,\tau)}{\tau}\right|_{\tau = t}\tag{11.20}
\]
and
\[
	\L(\x,t) = \left.\pd{\U_{(t)}(\x,\tau)}{\tau}\right|_{\tau = t} + \left.\pd{\R_{(t)}(\x,\tau)}{\tau}\right|_{\tau = t}\tag{11.21}
\]
are the symmetric and skew parts of $\L$, respectively.

Thus,
\[
	\D(\x,t) = \left.\pd{\U_{(t)}(\x,\tau)}{\tau}\right|_{\tau = t}\qquad and \qquad \W(\x,t) = \left.\pd{\R_{(t)}(\x,\tau)}{\tau}\right|_{\tau = t}\tag{11.22}
\]

\section{Material and Spatial Tensor Fields. Pullback and Pushforward Operations}
\subsection{Material and Spatial Tensor Fields}
\begin{enumerate}
	\item $\G$ is a \textbf{spatial tensor field} if, pointwise, $\G$ maps spatial vectors to spatial vectors
	\item $\G$ is a \textbf{material tensor field} if, pointwise, $\G$ maps material vectors to material vectors
	\item $\G$ is a \textbf{mixed tensor field} if, pointwise, $\G$ maps either spatial vectors to material vectors of material vectors to spatial vectors
\end{enumerate}

\begin{remark}
	$\V,\B,\L,\D,$ and $\W$ are spatial tensor fields\\
	$\U,\C,$ and $\E$ are material tensor fields\\
	$\F$ and $\R$ are mixed tensor fields
\end{remark}
\subsection{Pullback and Pushforward Operations}
A \textbf{pullback} of a spatial tensor field $\G$ is an operation which transforms $\G$ into a tensor mapping material to material.\\
EX:
\[
\mathbb{P}[\G]\eqdef\F^T\G\F\tag{12.1}
\]

EX:
\[
\underline{\mathbb{P}}[\G]\eqdef\F\inv\G\F^{-T}\tag{12.2}
\]

Note that these pullbacks preserve the operations sym and skw:
\[
	\sym\bb{P}][\G] = \bb{P}[\sym\G] \qquad and \qquad \skw\bb{P}[\G]=\bb{P}[\skw\G]
\]

Note that the following pullbacks preserve trace:
\[
	\F\inv\G\F \qquad and \qquad \F^T\G\F^{-T}
\]
Since pullbacks are linear transformations from tensors to tensors, they many be viewed as fourth-order tensors.\\
The following are pushforward operations:
\[
	\bb{P}\inv[\M] = \F^{-T}\M\F\inv \qquad and \qquad \underline{\bb{P}}\inv[\M] = \F\M\F^T \tag{12.5}
\]

An actual example is:
\[
	\bb{P}[\D] = \F^T\D\F = \dot{\E}
\]

\section{Modes of Evolution for Vector and Tensor Fields}
\subsection{Vector and Tensor Fields that Convect with the Body}
Recall, a spatial vector field $\f$ convects with the body if there is a time-independent material vector field $\f_R$ such that
\[
	\f(\x,t) = \F(\X,t)\f_R(\X)
\]
We will now call this \textbf{tangential convection}.
We will call the following \textbf{normal convection}:
\[
	\f(\x,t) = \F\invt(\X,t)\f_R(\X)
\]

From our definition, $\f$ convects as a tangent if
\[
	\dotbar{\F\inv\f} = 0
\]
Thus, $\f$ convects as a tangent if and only if it satisfies
\[
	\dot{\f}=\L\f\tag{13.4}
\]

Similarly, $\f$ convects as a normal if and only if it satisfies
\[
	\dot{\f} = -\L^T\f\tag{13.6}
\]
If $\f$ convects as a tangent and $\g$ convects as a normal, then 
\[
	\dotbar{\f\cdot\g} = 0\tag{13.7}
\]
\subsubsection{Tangentially Convecting Basis and its Dual Basis. Covariant and Contravariant Components of Spatial Fields}
Let $\lbrace \f_i\rbrace$ be a triad of spatial vectors fields which convect tangentially with associated material triad $\lbrace \m_i\rbrace$ such that 
\[
	\f_i = \F\m_i \qquad and \qquad \dot{m_i} = 0\tag{13.8}
\]
Then, $\lbrace\f_i\rbrace$ is a\textbf{tangentially convecting basis} and $\lbrace\m_i\rbrace$ is the associated \textbf{material basis}.
read to 109
\end{document}